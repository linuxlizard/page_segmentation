
%% bare_conf.tex
%% V1.3
%% 2007/01/11
%% by Michael Shell
%% See:
%% http://www.michaelshell.org/
%% for current contact information.
%%
%% This is a skeleton file demonstrating the use of IEEEtran.cls
%% (requires IEEEtran.cls version 1.7 or later) with an IEEE conference paper.
%%
%% Support sites:
%% http://www.michaelshell.org/tex/ieeetran/
%% http://www.ctan.org/tex-archive/macros/latex/contrib/IEEEtran/
%% and
%% http://www.ieee.org/

%%*************************************************************************
%% Legal Notice:
%% This code is offered as-is without any warranty either expressed or
%% implied; without even the implied warranty of MERCHANTABILITY or
%% FITNESS FOR A PARTICULAR PURPOSE! 
%% User assumes all risk.
%% In no event shall IEEE or any contributor to this code be liable for
%% any damages or losses, including, but not limited to, incidental,
%% consequential, or any other damages, resulting from the use or misuse
%% of any information contained here.
%%
%% All comments are the opinions of their respective authors and are not
%% necessarily endorsed by the IEEE.
%%
%% This work is distributed under the LaTeX Project Public License (LPPL)
%% ( http://www.latex-project.org/ ) version 1.3, and may be freely used,
%% distributed and modified. A copy of the LPPL, version 1.3, is included
%% in the base LaTeX documentation of all distributions of LaTeX released
%% 2003/12/01 or later.
%% Retain all contribution notices and credits.
%% ** Modified files should be clearly indicated as such, including  **
%% ** renaming them and changing author support contact information. **
%%
%% File list of work: IEEEtran.cls, IEEEtran_HOWTO.pdf, bare_adv.tex,
%%                    bare_conf.tex, bare_jrnl.tex, bare_jrnl_compsoc.tex
%%*************************************************************************

% *** Authors should verify (and, if needed, correct) their LaTeX system  ***
% *** with the testflow diagnostic prior to trusting their LaTeX platform ***
% *** with production work. IEEE's font choices can trigger bugs that do  ***
% *** not appear when using other class files.                            ***
% The testflow support page is at:
% http://www.michaelshell.org/tex/testflow/



% Note that the a4paper option is mainly intended so that authors in
% countries using A4 can easily print to A4 and see how their papers will
% look in print - the typesetting of the document will not typically be
% affected with changes in paper size (but the bottom and side margins will).
% Use the testflow package mentioned above to verify correct handling of
% both paper sizes by the user's LaTeX system.
%
% Also note that the "draftcls" or "draftclsnofoot", not "draft", option
% should be used if it is desired that the figures are to be displayed in
% draft mode.
%
\documentclass[conference]{IEEEtran}
% Add the compsoc option for Computer Society conferences.
%
% If IEEEtran.cls has not been installed into the LaTeX system files,
% manually specify the path to it like:
% \documentclass[conference]{../sty/IEEEtran}





% Some very useful LaTeX packages include:
% (uncomment the ones you want to load)


% *** MISC UTILITY PACKAGES ***
%
%\usepackage{ifpdf}
% Heiko Oberdiek's ifpdf.sty is very useful if you need conditional
% compilation based on whether the output is pdf or dvi.
% usage:
% \ifpdf
%   % pdf code
% \else
%   % dvi code
% \fi
% The latest version of ifpdf.sty can be obtained from:
% http://www.ctan.org/tex-archive/macros/latex/contrib/oberdiek/
% Also, note that IEEEtran.cls V1.7 and later provides a builtin
% \ifCLASSINFOpdf conditional that works the same way.
% When switching from latex to pdflatex and vice-versa, the compiler may
% have to be run twice to clear warning/error messages.






% *** CITATION PACKAGES ***
%
\usepackage{cite}
% cite.sty was written by Donald Arseneau
% V1.6 and later of IEEEtran pre-defines the format of the cite.sty package
% \cite{} output to follow that of IEEE. Loading the cite package will
% result in citation numbers being automatically sorted and properly
% "compressed/ranged". e.g., [1], [9], [2], [7], [5], [6] without using
% cite.sty will become [1], [2], [5]--[7], [9] using cite.sty. cite.sty's
% \cite will automatically add leading space, if needed. Use cite.sty's
% noadjust option (cite.sty V3.8 and later) if you want to turn this off.
% cite.sty is already installed on most LaTeX systems. Be sure and use
% version 4.0 (2003-05-27) and later if using hyperref.sty. cite.sty does
% not currently provide for hyperlinked citations.
% The latest version can be obtained at:
% http://www.ctan.org/tex-archive/macros/latex/contrib/cite/
% The documentation is contained in the cite.sty file itself.






% *** GRAPHICS RELATED PACKAGES ***
%
\usepackage{graphicx}
% \ifCLASSINFOpdf
%  \usepackage[pdftex]{graphicx}
%  % declare the path(s) where your graphic files are
%  \graphicspath{{../}}
%  % and their extensions so you won't have to specify these with
%  % every instance of \includegraphics
%  \DeclareGraphicsExtensions{.pdf,.jpeg,.png,.png}
%%  \DeclareGraphicsExtensions{.png}
%\else
%  % or other class option (dvipsone, dvipdf, if not using dvips). graphicx
%  % will default to the driver specified in the system graphics.cfg if no
%  % driver is specified.
%  \usepackage[dvips]{graphicx}
%  % declare the path(s) where your graphic files are
%  \graphicspath{{../}}
%  % and their extensions so you won't have to specify these with
%  % every instance of \includegraphics
%  \DeclareGraphicsExtensions{.png}
%\fi
% graphicx was written by David Carlisle and Sebastian Rahtz. It is
% required if you want graphics, photos, etc. graphicx.sty is already
% installed on most LaTeX systems. The latest version and documentation can
% be obtained at: 
% http://www.ctan.org/tex-archive/macros/latex/required/graphics/
% Another good source of documentation is "Using Imported Graphics in
% LaTeX2e" by Keith Reckdahl which can be found as epslatex.ps or
% epslatex.pdf at: http://www.ctan.org/tex-archive/info/
%
% latex, and pdflatex in dvi mode, support graphics in encapsulated
% postscript (.png) format. pdflatex in pdf mode supports graphics
% in .pdf, .jpeg, .png and .mps (metapost) formats. Users should ensure
% that all non-photo figures use a vector format (.png, .pdf, .mps) and
% not a bitmapped formats (.jpeg, .png). IEEE frowns on bitmapped formats
% which can result in "jaggedy"/blurry rendering of lines and letters as
% well as large increases in file sizes.
%
% You can find documentation about the pdfTeX application at:
% http://www.tug.org/applications/pdftex





% *** MATH PACKAGES ***
%
%\usepackage[cmex10]{amsmath}
% A popular package from the American Mathematical Society that provides
% many useful and powerful commands for dealing with mathematics. If using
% it, be sure to load this package with the cmex10 option to ensure that
% only type 1 fonts will utilized at all point sizes. Without this option,
% it is possible that some math symbols, particularly those within
% footnotes, will be rendered in bitmap form which will result in a
% document that can not be IEEE Xplore compliant!
%
% Also, note that the amsmath package sets \interdisplaylinepenalty to 10000
% thus preventing page breaks from occurring within multiline equations. Use:
%\interdisplaylinepenalty=2500
% after loading amsmath to restore such page breaks as IEEEtran.cls normally
% does. amsmath.sty is already installed on most LaTeX systems. The latest
% version and documentation can be obtained at:
% http://www.ctan.org/tex-archive/macros/latex/required/amslatex/math/





% *** SPECIALIZED LIST PACKAGES ***
%
\usepackage{algorithmic}
% algorithmic.sty was written by Peter Williams and Rogerio Brito.
% This package provides an algorithmic environment fo describing algorithms.
% You can use the algorithmic environment in-text or within a figure
% environment to provide for a floating algorithm. Do NOT use the algorithm
% floating environment provided by algorithm.sty (by the same authors) or
% algorithm2e.sty (by Christophe Fiorio) as IEEE does not use dedicated
% algorithm float types and packages that provide these will not provide
% correct IEEE style captions. The latest version and documentation of
% algorithmic.sty can be obtained at:
% http://www.ctan.org/tex-archive/macros/latex/contrib/algorithms/
% There is also a support site at:
% http://algorithms.berlios.de/index.html
% Also of interest may be the (relatively newer and more customizable)
% algorithmicx.sty package by Szasz Janos:
% http://www.ctan.org/tex-archive/macros/latex/contrib/algorithmicx/




% *** ALIGNMENT PACKAGES ***
%
%\usepackage{array}
% Frank Mittelbach's and David Carlisle's array.sty patches and improves
% the standard LaTeX2e array and tabular environments to provide better
% appearance and additional user controls. As the default LaTeX2e table
% generation code is lacking to the point of almost being broken with
% respect to the quality of the end results, all users are strongly
% advised to use an enhanced (at the very least that provided by array.sty)
% set of table tools. array.sty is already installed on most systems. The
% latest version and documentation can be obtained at:
% http://www.ctan.org/tex-archive/macros/latex/required/tools/


%\usepackage{mdwmath}
%\usepackage{mdwtab}
% Also highly recommended is Mark Wooding's extremely powerful MDW tools,
% especially mdwmath.sty and mdwtab.sty which are used to format equations
% and tables, respectively. The MDWtools set is already installed on most
% LaTeX systems. The lastest version and documentation is available at:
% http://www.ctan.org/tex-archive/macros/latex/contrib/mdwtools/


% IEEEtran contains the IEEEeqnarray family of commands that can be used to
% generate multiline equations as well as matrices, tables, etc., of high
% quality.


%\usepackage{eqparbox}
% Also of notable interest is Scott Pakin's eqparbox package for creating
% (automatically sized) equal width boxes - aka "natural width parboxes".
% Available at:
% http://www.ctan.org/tex-archive/macros/latex/contrib/eqparbox/





% *** SUBFIGURE PACKAGES ***
\usepackage{subfigure}
%\usepackage[tight,footnotesize]{subfigure}
% subfigure.sty was written by Steven Douglas Cochran. This package makes it
% easy to put subfigures in your figures. e.g., "Figure 1a and 1b". For IEEE
% work, it is a good idea to load it with the tight package option to reduce
% the amount of white space around the subfigures. subfigure.sty is already
% installed on most LaTeX systems. The latest version and documentation can
% be obtained at:
% http://www.ctan.org/tex-archive/obsolete/macros/latex/contrib/subfigure/
% subfigure.sty has been superceeded by subfig.sty.



%\usepackage[caption=false]{caption}
%\usepackage[font=footnotesize]{subfig}
% subfig.sty, also written by Steven Douglas Cochran, is the modern
% replacement for subfigure.sty. However, subfig.sty requires and
% automatically loads Axel Sommerfeldt's caption.sty which will override
% IEEEtran.cls handling of captions and this will result in nonIEEE style
% figure/table captions. To prevent this problem, be sure and preload
% caption.sty with its "caption=false" package option. This is will preserve
% IEEEtran.cls handing of captions. Version 1.3 (2005/06/28) and later 
% (recommended due to many improvements over 1.2) of subfig.sty supports
% the caption=false option directly:
%\usepackage[caption=false,font=footnotesize]{subfig}
%
% The latest version and documentation can be obtained at:
% http://www.ctan.org/tex-archive/macros/latex/contrib/subfig/
% The latest version and documentation of caption.sty can be obtained at:
% http://www.ctan.org/tex-archive/macros/latex/contrib/caption/




% *** FLOAT PACKAGES ***
%
%\usepackage{fixltx2e}
% fixltx2e, the successor to the earlier fix2col.sty, was written by
% Frank Mittelbach and David Carlisle. This package corrects a few problems
% in the LaTeX2e kernel, the most notable of which is that in current
% LaTeX2e releases, the ordering of single and double column floats is not
% guaranteed to be preserved. Thus, an unpatched LaTeX2e can allow a
% single column figure to be placed prior to an earlier double column
% figure. The latest version and documentation can be found at:
% http://www.ctan.org/tex-archive/macros/latex/base/



%\usepackage{stfloats}
% stfloats.sty was written by Sigitas Tolusis. This package gives LaTeX2e
% the ability to do double column floats at the bottom of the page as well
% as the top. (e.g., "\begin{figure*}[!b]" is not normally possible in
% LaTeX2e). It also provides a command:
%\fnbelowfloat
% to enable the placement of footnotes below bottom floats (the standard
% LaTeX2e kernel puts them above bottom floats). This is an invasive package
% which rewrites many portions of the LaTeX2e float routines. It may not work
% with other packages that modify the LaTeX2e float routines. The latest
% version and documentation can be obtained at:
% http://www.ctan.org/tex-archive/macros/latex/contrib/sttools/
% Documentation is contained in the stfloats.sty comments as well as in the
% presfull.pdf file. Do not use the stfloats baselinefloat ability as IEEE
% does not allow \baselineskip to stretch. Authors submitting work to the
% IEEE should note that IEEE rarely uses double column equations and
% that authors should try to avoid such use. Do not be tempted to use the
% cuted.sty or midfloat.sty packages (also by Sigitas Tolusis) as IEEE does
% not format its papers in such ways.





% *** PDF, URL AND HYPERLINK PACKAGES ***
%
%\usepackage{url}
% url.sty was written by Donald Arseneau. It provides better support for
% handling and breaking URLs. url.sty is already installed on most LaTeX
% systems. The latest version can be obtained at:
% http://www.ctan.org/tex-archive/macros/latex/contrib/misc/
% Read the url.sty source comments for usage information. Basically,
% \url{my_url_here}.





% *** Do not adjust lengths that control margins, column widths, etc. ***
% *** Do not use packages that alter fonts (such as pslatex).         ***
% There should be no need to do such things with IEEEtran.cls V1.6 and later.
% (Unless specifically asked to do so by the journal or conference you plan
% to submit to, of course. )


% correct bad hyphenation here
\hyphenation{op-tical net-works semi-conduc-tor}


\begin{document}
%
% paper title
% can use linebreaks \\ within to get better formatting as desired
\title{Page Segmentation Performance Using Horizontal Image Strips Instead of Full Page Images}


% author names and affiliations
% use a multiple column layout for up to three different
% affiliations
\author{\IEEEauthorblockN{David Poole}
\IEEEauthorblockA{Marvell Semiconductor Inc.
\\
Boise, ID  USA \\
Email: dpoole@marvell.com}
\and
\IEEEauthorblockN{Dr. Elisa Barney Smith}
\IEEEauthorblockA{Boise State University\\
Boise, ID  USA \\
Email: EBarneySmith@BoiseState.edu}
}

% conference papers do not typically use \thanks and this command
% is locked out in conference mode. If really needed, such as for
% the acknowledgment of grants, issue a \IEEEoverridecommandlockouts
% after \documentclass

% for over three affiliations, or if they all won't fit within the width
% of the page, use this alternative format:
% 
%\author{\IEEEauthorblockN{Michael Shell\IEEEauthorrefmark{1},
%Homer Simpson\IEEEauthorrefmark{2},
%James Kirk\IEEEauthorrefmark{3}, 
%Montgomery Scott\IEEEauthorrefmark{3} and
%Eldon Tyrell\IEEEauthorrefmark{4}}
%\IEEEauthorblockA{\IEEEauthorrefmark{1}School of Electrical and Computer Engineering\\
%Georgia Institute of Technology,
%Atlanta, Georgia 30332--0250\\ Email: see http://www.michaelshell.org/contact.html}
%\IEEEauthorblockA{\IEEEauthorrefmark{2}Twentieth Century Fox, Springfield, USA\\
%Email: homer@thesimpsons.com}
%\IEEEauthorblockA{\IEEEauthorrefmark{3}Starfleet Academy, San Francisco, California 96678-2391\\
%Telephone: (800) 555--1212, Fax: (888) 555--1212}
%\IEEEauthorblockA{\IEEEauthorrefmark{4}Tyrell Inc., 123 Replicant Street, Los Angeles, California 90210--4321}}




% use for special paper notices
%\IEEEspecialpapernotice{(Invited Paper)}




% make the title area
\maketitle


% ========================================
%  Abstract
% ========================================

\begin{abstract}
%\boldmath
Most page segmentation algorithms are geared toward full page analysis.
However, in resource constrained environments, 
such as a dedicated consumer desktop scanner or MFP (Multi-Function
Printer), 
a full page is not available. 
Storing a full page would require a cost prohibitive amount of
memory. 

We study two page segmentation algorithms (Voronoi and RAST) to discover
how each algorithm behaves in the full page environment relative to  a smaller image subset. The OCRopus page
segmentation toolkit is used as the basis of this work. 
The tests are run against a set of scanned document images with known ground
truth text/non-text zones.  The input images are divided into strips. The full
page segmentation performance of OCRopus is measured. The results are then
compared against the OCRopus performance with each set of smaller strips. 

Our evaluation shows RAST performs well in an image strip page
segmentation. The results are encouraging enough to study modifications to RAST
to improve its performance in an image strip page segmentation.  
\end{abstract}

% IEEEtran.cls defaults to using nonbold math in the Abstract.
% This preserves the distinction between vectors and scalars. However,
% if the conference you are submitting to favors bold math in the abstract,
% then you can use LaTeX's standard command \boldmath at the very start
% of the abstract to achieve this. Many IEEE journals/conferences frown on
% math in the abstract anyway.

% no keywords




% For peer review papers, you can put extra information on the cover
% page as needed:
% \ifCLASSOPTIONpeerreview
% \begin{center} \bfseries EDICS Category: 3-BBND \end{center}
% \fi
%
% For peerreview papers, this IEEEtran command inserts a page break and
% creates the second title. It will be ignored for other modes.
\IEEEpeerreviewmaketitle


% ========================================
%  Introduction
% ========================================

\section{Introduction}
% no \IEEEPARstart

High image quality is a trait desired in all document images.
In the printer environment this can be increased by differentiating the processing of text versus image zones.
Being able to differentiate a text region from a graphics region
will allow the firmware/hardware to optimize the output. A text region should
use a heavy contrast (clip to black/white) and sharpen. A graphics region
should preserve as much dynamic range as possible. A smooth or even a halftone
descreen could also be performed on a graphics region.
A page can be divided into
zones of different content such as text paragraph, business graphic, halftoned
image, etc. through page segmentation.

Page segmentation algorithms are usually designed under the assumption that a full page image will be provided.
In the consumer market, a scanner/MFP (Multi-Function printer=a scanner+printer combination) is a commodity and, as such, cost is the driving factor. 
In a consumer level scanner or MFP a full image, even at low resolution, exceeds available memory. 
A full 300 dpi monochrome image is 8M. A color scan/copy is
25M. At higher qualities, a 600 dpi scan would be over 100M. A typical consumer
MFP has between 8M and 128M. 
Therefore a page segmentation algorithm that performs well on a small portion, or strip, of a page image is needed so these devices can use content based preprocessing.

In addition to MFPs
strip based page segmentation would have application 
in other areas where a full page image could be too large to
store in available memory, such as a micro-fiche scanner. 
With the proliferation of mobile devices (phones, tablets, etc.), camera-based
imaging has become a topic of interest.  However, mobile devices are memory
constrained in much the same way as desktop scanner/copier products. 
Processing on small strips could be performed as described in this paper,
rather than storing an entire image in active memory.

This paper explores the behavior of two page segmentation algorithms on image strips.
An image strip is the full-page width, but a fraction of the height of the original image.
Most page segmentation algorithmns are tested against a full-page image. 
The strip results are compared to the full-page results. The goal of the
experiment is to find a page segmentation algorithm that best scales downward
to image strips. The best algorithm would be the topic of further study toward
tuning the algorithm for page segmentation with image strips.

% ========================================
%  Previous Work
% ========================================

\section{Previous Work}

%There is a large body of literature on the problem of page segmentation with
%several competing algorithms. There is also a body of literature studying the
%relative merits of the different algorithms. [Performance Comparison of Six
%Algorithms] highlighted six algorithms: XYCUT, SMEARING, VORONOI, WHITESPACE
%ANALYSIS, and CONSTRAINED TEXT LINE DETECTION. 

There are many page segmentation algorithms.  Page segmentation algorithms are
usually divided into top-down and bottom-up categories
\cite{shafait2006performance,mao2000empirical} with some algorithms considered
a hybrid mix of the two.
A top-down algorithm examines the entire image, finding
structure in the image, then drilling into deeper detail.  
A bottom-up algorithm examines the image at the pixel level then builds a
larger picture of the document from smaller components.  A good description of
top-down vs. bottom-up segmentation algorithms is
available at \cite{baird1994background}.

Examples of top-down algorithms are 
    X-Y Cut, \cite{nagy1992prototype},
    Whitespace Analysis \cite{baird1994background}, and
    Constrained Text Line Detection \cite{breuel2002two}.
X-Y Cut recursively breaks an image into
rectangular blocks using horizontal and vertical projection histograms.
Rectangles are recursively split based on valleys in the histogram.  Whitespace
analysis starts with bounding boxes of black
connected components. Rectangles of whitespace are built around the black
components such that no rectangle intersects with a black component. The
resulting rectangles are repeatedly combined to form a cover. The bounding box
of the uncovered region is treated as a text segment.  Constrained Text Line
Detection first finds whitespace rectangles that represent gutters.
The gutters become obstacles in computing the bounding boxes of text lines.

Examples of bottom-up algorithms are 
    Docstrum \cite{o1993document},
    Smearing \cite{wong1982document}, and
    Voronoi \cite{kise1998segmentation}.
Docstrum uses the distance between clusters of connected components. The
k-nearest-neighbors of the components are measured and the docstrum is the plot
of all nearest neighbor pairs on a page.  Smearing transforms the image at the pixel level
by setting 0 pixels to 1 based on their neighbors' values. Black areas are
connected together. The algorithm is run along rows then along columns and the
two images combined.
The Voronoi segmentation starts with centroids of connected components. A point voronoi
diagram is built around those points. Extraneous edges are filtered out leaving
an area Voronoi diagram of island regions.

Baird \cite{baird2006versatile} has a novel and interesting approach to page
segmentation. Each pixel is classified by around 100 features. Thousands of
images are used to build a training set. K-Nearest Neighbors is used
to classify each pixel in a test document. As one of the admitted assumptions
in the paper is enormous computing resources, this algorithm would be
inappropriate for the resource constraints targeted in this paper.

RAST is not part of the performance comparison in \cite{shafait2006performance,mao2000empirical}. 
RAST is a bottom-up algorithm that starts with bounding boxes of connected
components. Statistics on the boxes are used to decide if the box contains a
character. Next, the whitespace cover is computed, similar to the Constrained
Text Line algorithm \cite{breuel2002two}.  As both RAST and the Constrained
Text Line algorithm are by the same author, RAST builds upon the Constrained
Text Line algorithm. The text lines are found based on contiguous character
boxes.

Voronoi was judged to be of high quality in \cite{shafait2006performance}. The
RAST algorithm was not included in the \cite{shafait2006performance} comparison
but was judged to have high quality in \cite{winder2010extending} and is one of
the core algorithms in OCRopus \cite{breuel2008ocropus}. 

In this paper Voronoi and RAST are compared for their potential for use in a
memory constrained environment. A third algorithm will be included before the
final publication date.

% ===================================
%  Describing the Performance Metrics
% ===================================

\section{Performance Metrics}
%``To quantitatively evaluate algorithms, a meaningful and computable metric is
%essential.'' \cite{phillips1999empirical}
The most referenced paper studying multiple page segmentation is
\cite{shafait2006performance} which used the performance metric as defined in
\cite{phillips1999empirical}. Another method used in other works the Page Segmentation
Evaluation Toolkit (PSET) \cite{mao2000pset,mao2002software}.

The text area matching algorithm of \cite{phillips1999empirical} calculates a
text area matching score by first calculating the intersection of the ground
truth bounding box and the result bounding box. The intersection is
then divided by the larger of the two areas. A set of metrics on how well the
zones were discovered are next calculated.  The one-to-many and many-to-one
matches are calculated both from the perspective of the ground truth and the
detected areas. 

For example, a single ground truth bounding box could have been split into
multiple boxes in the result. That would be a one-to-many error. Multiple
ground truth boxes combined into a single box would be a many-to-one error.

The metric chosen in this paper is based on finding only text/non-text zones.
The UW-III ZONEBOX files were particularly well adapted to this simple
metric. The \cite{winder2010extending} test image set was divided into bounding
boxes for text and non-text zones.  The metric itself was calculated using the
zone comparison program written as part of \cite{winder2010extending}. 

%The Page Segmentation Evaluation Toolkit (PSET) \cite{mao2000pset} was used to analyzed XYCUT,
%DOCSTRUM, and VORONOI algorithms as well as two commercial systems from
%Caere Corporation and ScanSoft Corporation. \cite{mao2003document} not only
%cataloged many segmentation algorithms but their performance as well.

The weights used in the zone comparison program are shown in Table
\ref{table:table-weights}. The correct one-to-one matches are weighted more
heavily than the one-to-many and the many-to-one matches, i.e., the correct
matches are weighed more heavily than the wrong matches.

\begin{table}
\begin{center}
\caption{Weights used in the zone comparison program}
\label{table:table-weights}
\label{table-weights}
\begin{tabular}{|c|c|c|c|c|c|}
\hline
  w1 & w2 & w3 & w4 & w5 & w6 \\
 \hline
  1.0  & 0.75 & 0.75 & 1.0 & 0.75 & 0.75 \\
\hline
\end{tabular}
\end{center}
\end{table}

% ========================================
%  Describing the Method of the Experiments
% ========================================

\section{Method}

To quantitatively compare page segmentation algorithms, a simplified version of
the text area comparison algorithm from \cite{phillips1999empirical} is used.
Metrics of bounding boxes of zones, measurements such as
horizontally/vertically merged lines, false detections, etc. are combined to
form a Text-line Accuracy.  The zone comparison program is based on the
metric chosen for the ICDAR 2007 Handwriting Segmentation Contest and is
described in \cite{antonacopoulos2007page}.

%\cite{mao2000pset} is used in this paper to compare the performance of Voronoi
%and RAST on page and strips across two data sets containing several categories
%for page images.

%Two page segmentation algorithms, Voronoi and RAST, are compared in this paper.
%The algorithms are based on the OCRopus v0.4.3  \cite{breuel2008ocropus}, but
%include the enhancements reported in \cite{winder2011icdar}. This improved
%their performance by @@insert quantitative or qualitiative info@@.  To be able
%to compare their outputs with the ground truth, \cite{breuel2008ocropus} was
%expanded to output the zone areas in XML.

In \cite{winder2010extending} the OCRopus v0.4.3 was extended with
additional page segmentation capabilities. The RAST algorithm was improved by
classifying and merging graphics bounding boxes and handling text boxes
overlapping graphics.  The Voronoi algorithm in OCRopus did not support zone
classification and that work was completed by Winder.

The Winder work improved average RAST performance across all image classes by
25\%. The original Voronoi implementation did not include zone classification
but the overall Voronoi performance was not as good as RAST. Where RAST would
segment at 100\%, Voronoi would get around 80\%.

%XYCut and RAST were extended to add XML zone output. The XML zone output
%allowed measurement of the improvement in page segmentation.  This study uses
%the XML output from \cite{winder2010extending} from the RAST and Voronoi page
%segmentation programs. 

The RAST and Voronoi page segmentation programs read a PNG file and output a
segmented PNG image and an XML zone file. The metric comparison is the same as
used in \cite{winder2010extending} which is the same as used in
\cite{phillips1999empirical}\cite{antonacopoulos2007page}. The metric comparison
reads a ground truth XML zone file and an output XML file. A metric with values
in the range $[0, 1.0]$ is output where 0 indicates no match and 1.0 indicates
full match. 

The ground truth pages used in this study are from the UW-III dataset set of
1600 “LINEWORD” pages. Also used were 91 ground truth images used in \cite{winder2010extending}.
The Winder data set is much cleaner than the UW-III dataset
and contain simple high quality scanned pages consisting of one or two
columns of text, some pages interspersed with images. Tables \ref{table:Winder Image Classes} 
and \ref{table:UW-III Image Classes} describe the contents of these datasets
and the categories of page images they contain.  They list the image classes 
and an abbreviation (used in the figures) and a short description of the
classes is also included.

The \cite{winder2010extending} consists of different clases. Most of the images
are scans of pages designed with a specific function. For example, the "Single
Column" pages are a full page of text paragraphs. Only the "Magazine" and
"Double Column Pictures Scientific" are not generated pages. The "Magazine"
pages are scans from actual magazines and contain mixed layout text with
halftoned graphics. The "Double Column Pictures Scientific" are scans of
scholarly papers.  As they are scans of individual pieces of paper, the
document image quality is quite good.

The UW-III LINEWORD dataset are scans of journals or scans of photocopies of
journals. The images are of highly varying quality. Many of the scans have page
shadow (dark area as the page fold rises above the scanner glass). Several of
the scans have partial images of opposing pages. The UW-III dataset is a
difficult segmentation challenge.

All images tested were 300 dpi.  The pages were divided into strips of 1" or 2"
(300 rows and 600 rows). A window slides down the page image, 10 rows at a time
(for 300) or 20 rows (for 600).  The sliding window simulates the scanned image
in memory. As new rows arrive, the oldest rows are ejected.  Each 
image was split into 330 PNG files of 300 rows each and 165 PNG files of 600
rows each.  

%\begin{tabular}
%\begin{tabular}{|r|l|}
%\hline
%    1 inch  & 300 rows  & 780kB mono  & 2,34MB RGB color \\
%    2 inches  & 600 rows  & 1,56MB mono  & 4,68MB RGB color \\
%    11 inches & 3300 rows & 8,5M mono & 25,2MB RGB color \\
%%    1 inch  & 300 rows  & 780kB mono  & 2.34MB RGB color \\
%%    2 inches  & 600 rows  & 1.56MB mono  & 4.68MB RGB color \\
%%    11 inches & 3300 rows & 8.5M mono & 25.2MB RGB color \\
%\hline
%\label{table:image-size}
%\end{tabular}

The UW-III images contain ZONEBOX “.box” files indicating ground truth text
areas of the full page images. The Winder images are accompanied by XML files
of ground truth zoneboxes of both text and non-text. The UW-III .box files only
contained text areas.

In order to compare the segmentation results from the strip images against a
ground truth for pages, the ground truth files are divided into strips.  A full
page width strip rectangle of 300 or 600 rows was intersected with the
zoneboxes. The intersection of the two rectangles became new ground truth
zoneboxes. The page width strip was repeatedly slid 10 or 20 rows down then the
intersection was tested again. The final output of each “slide” is a .XML file
indicating the zoneboxes in each window.

The \cite{winder2010extending} RAST and Voronoi page segmentation programs were
run against the full page to create a behavior baseline.  The RAST and Voronoi
page segmentation programs were then run against the new strip images. The output XML
was saved to a separate file. The \cite{winder2010extending} zone comparison
program was run against the sliced ground truth file and the experimental
results. 

%\section{RAST and Voronoi with OCRopus}
%While the algorithms used in this paper were RAST and Voronoi, the code that
%did the actual page segmentation was two separate executables, both based on on
%OCRopus 4.3.2. The two programs instantiated an instance of the OCRopus
%segmentation class, SegmentPageByRAST() and SegmentPageByVoronoi(),
%respectively. 

%\subsection{RAST}
%
%\subsection{Voronoi}

\begin{table}
\caption{Data from Winder Dataset}
\label{table:Winder Image Classes}
\begin{tabular}{|l|l|l|l|}
\hline
\textit{\textbf{Image Class}} & \textit{\textbf{Abbrev.}} & \textit{\textbf{Description}} & \textit{\textbf{Pages}}  \\ 
\hline
\hline
    Single Column & SC & Single column of only text &  10  \\ 
    \hline
    Single Column & SCP & Single column of & 10 \\
    with Pictures &     & text with picture intermixed &    \\ 
    \hline

    Mixed Column & MC & Text in one or two columns & 10 \\
    \hline

    Mixed Column & MCP & Text in one or two columns & 10 \\
    with Pictures&    & with picture intermixed & \\

    \hline
    Double Column & DC & Two columns of only text & 10 \\  
    \hline
    Double Column & DCP & Two columns of text with & 10 \\ 
    with Pictures &     & picture intermixed &    \\  
    \hline
    Double Column  & DCS & Two columns of text & 21 \\
    with Scientific &     &  with technical graphic  & \\ 
    Pictures &  & & \\ 
    \hline
    Magazine & M & Scan of magazine page; & 10 \\
             &   & Graphics with text flowing & \\
             &   & around the images & \\
\hline
\end{tabular}
\end{table}

\begin{table}
\caption{UW-III Image Classes}
\label{table:UW-III Image Classes}
\begin{tabular}{|l|l|l|}
\hline
\textit{\textbf{Image Class}} & \textit{\textbf{Description}} & \textit{\textbf{Pages}}  \\ 
\textit{\textbf{Abbrev.}} & & \\ 
\hline
    \hline
    A, C, D,   & Scans of first generation & 734 \\
    E, H, I,   & English journal photocopies & \\ 
    J, K, V    &  & \\ 
    \hline
    N & Scans from English Journal photocopies & 119 \\ 
      & (supplied by UNLV) & \\ 
    \hline
    S & Direct scans from English Journal pages & 125 \\ 
    \hline
    W & Binary scans from original English Journals & 622 \\
      & and 1st and 2nd generation English journal & \\
      & photocopies  & \\ 
    \hline
\end{tabular}
\end{table}

% ====================================
%          Results
% ====================================
\section{Results}

The goal of this research is to find the algorithm that will best maintain the
performance of segmenting a full page image when the image is shrunk to a small
strip.  All images from both the Winder and the UW-III datasets with 
the 300, 600 row strips, and the fullpage were run with both the RAST and
Voronoi algorithms. 
Comparing the results of segmentation strip size vs. algorithm per dataset
is shown in Figure~\ref{fig:all-qualitative}. 
The Winder data set is much cleaner than the UW-III dataset 
so the better results for Winder vs. UW-III are no surprise. 
The RAST algorithm appears to perform much better than Voronoi on both data sets.
RAST on a 300 row vs. a 600 row strip in the Winder dataset is little changed.
300 and 600 performed almost identically, but not as well as a full page. There
was very little additional improvement in going from a 300 row strip to a 600
row strip which would indicate the smaller strip size (and corresponding lower
memory requirements) work well.

% All 300,600,full rast,vor winder,uwiii on one graph
\begin{figure}[all-qualitative]
\begin{center}
\caption{UW-III and Winder Segmentation Results}
\label{fig:all-qualitative}
  \includegraphics[width=6cm]{images/all_qual.png}
\end{center}
\end{figure}

The UW-III results are more puzzling. The degradation of results from the
strip images to the full page is understandable given the extra challenges in
the UW-III dataset images. When the image was broken up into horizontal strips,
we avoided some of the page scanning artifacts. The UW-III Voronoi results are
show an increase in performance from 300 to 600 then a sharp drop. That result
needs further exploration.

%Overall, the results are very encouraging. The segmentation algorithms
%performed better on the image strips than the full page. There were several 1.0
%(perfect match) image strip results. A perfect match would usually occur with
%pages with little page edge noise and clearly aligned blocks of text. Because
%the 300 and 600 behaved so closely, all the following results only examine the
%300 stripsize. 

% Winder 2x2 all results
\begin{figure}[winder_2x2]
\begin{center}
\caption{Winder All Results}
\label{fig:winder_2x2}
  \includegraphics[width=6cm]{images/winder_2x2.png}
\end{center}
\end{figure}

% UW-III 2x2 all results
\begin{figure}[uwiii_2x2]
\begin{center}
\caption{UW-III All Results}
\label{fig:uwiii_2x2}
  \includegraphics[width=6cm]{images/uwiii_2x2.png}
\end{center}
\end{figure}

Figures~\ref{fig:winder_2x2} and \ref{fig:uwiii_2x2} are histograms of the
metrics across all results for the two datasets. Winder has excellent
performance on fullpage RAST and corresponding excellent results with 300 RAST.
The Voronoi results are not as good as RAST. UW-III show Voronoi performing
better than RAST on full pages. However, RAST 300 has much better performance
than RAST fullpage.

% Winder by class 
\begin{figure}[winder-class-rast-vs-vor]
\caption{Winder Fullpage RAST vs Voronoi by Class}
\begin{center}
\includegraphics[width=6.0cm]{images/winder_class_rast_vs_vor.png}
\label{fig:winder-class-rast-vs-vor}
\end{center}
\end{figure}

% UW-III by class 
\begin{figure}[uwiii-class-rast-vs-vor]
\caption{UW-III Fullpage RAST vs Voronoi by Class}
\begin{center}
\includegraphics[width=6.0cm]{images/uwiii_class_rast_vs_vor.png}
\label{fig:uwiii-class-rast-vs-vor}
\end{center}
\end{figure}

%\begin{figure}[strip-image-showing-perfect-hit]
%\caption{Strip Image with Perfect Zone Discovery}
%\label{fig:strip-image-showing-perfect-hit}
%\includegraphics[width=8.0cm]{images/A00AZONE_300_010_0360_zones.png}
%\end{figure}

Figures~\ref{fig:winder-class-rast-vs-vor} and \ref{fig:uwiii-class-rast-vs-vor} show all results of 
the fullpage RAST and Voronoi alongside the results
for stripsize 300 divided by image class in Tables~\ref{table:Winder Image Classes} 
and \ref{table:UW-III Image Classes}. In the Winder results the fullpage RAST performs the best in
all cases except Magazine (M). 
The Magazine images
contain text flowing around images, sometimes non-rectangular. Voronoi would be
better at handling the freer flowing format than the block-oriented RAST.
The encouraging result is the RAST 300 is in
close second to the fullpage on several classes. The Multi-Column layout (MC)
result is almost identical. Given the restricted size of the segmentation area,
the RAST 300 is performing quite well. The UW-III results paint a slightly
different view. On several of the classes, the Voronoi performs better than the
RAST. However, the RAST 300 performs as good as or better than all other
contenders in ten of the twelve tests!

The metric used to calculate strip match quality would yield a 1.0 on a perfect
match, when the ground truth and the result segmentation perfectly matched. However, there
were several reasons a zero metric would arise. A zero could result from an
empty strip--there is nothing to match and the ground truth also would be
empty. The resulting equation would be 0/0 giving NAN. Because the original
metric was not designed to handle empty images (why would the test set contain
an empty page?), the metric would give a zero on both an empty image/ground
truth and a miss (no matching zones). The metric measurement program did not
expect an empty zone box. Consequently, the zeros are not shown in the above
graphs.

Before the final publication date, work will include matching empty ground
truth with an empty segmented strip. Preliminary examination shows 11\% of the
UW-III strips are empty (mostly at the bottom of pages).

%
% Table of Zero results; created with numcrunch.py
%

%\begin{table}
%\begin{center}
%\caption{Results of Page Segmentation Tests}
%    \begin{tabular}{| l | r | r | c | }
%    \hline
%    Name    &   Total   & Zeros & \% \\
%            &   Metrics &       & Non Zero \\
%    \hline 
%    \hline 
%    Winder RAST 300 & 50 283 & 5 159 & 0.81 \\
%    \hline
%
%    Winder RAST 600 & 25 831 & 2 253 & 0.84 \\
%    \hline
%
%    Winder RAST fullpage & 179 & 3 & 0.97 \\
%    \hline
%
%    Winder Vor 300 & 42 143 & 13 299 & 0.52 \\
%    \hline
%
%    Winder Vor 600 & 22 802 & 5 282 & 0.62 \\
%    \hline
%
%    Winder Vor fullpage & 181 & 1 & 0.99 \\
%    \hline
%
%    UWIII RAST 300 & 867 596 & 189 060 & 0.64 \\
%    \hline
%
%    UWIII RAST 600 & 444 353 & 84 103 & 0.68 \\
%    \hline
%
%    UWIII RAST fullpage & 2 939 & 511 & 0.70 \\
%    \hline
%
%    UWIII Vor 300 & 765 845 & 290 811 & 0.45 \\
%    \hline
%
%    UWIII Vor 600 & 408 361 & 120 095 & 0.55 \\
%    \hline
%
%    UWIII Vor fullpage & 2565 & 885 & 0.49 \\
%    \hline
%
%    \end{tabular}
%\end{center}
%\end{table}

%
% Discussion of Winder Results
%
\subsection{Strip Performance}

Figure~\ref{fig:winder-300-double-pic-2col300} shows the performance strip by
strip on one somewhat complicated page of two columns with included halftoned
graphics.  The overall RAST performance was 0.78 on the full page image, while
the mean of the sliding window was 0.59.  Note in many cases the performance of
an individual strip is better than the strips' mean over the whole page and
even above the performance of the entire page.  Again, several zeros were also
found where the page itself was empty so the strip was blank. The performance
drop from 0.78 to 0.59 could be attribed to the zeros at the bottom third (a
strong graphics location) and the bottom of the page (whitespace).

\begin{figure}[winder-300-double-pic-2col300]
\includegraphics[width=6.0cm]{images/winder_full_rast_double_pic_2col300_2.png}
\caption{Winder 300 Strips vs Full Page RAST Two Columns with Picture}
\label{fig:winder-300-double-pic-2col300}
\end{figure}

%The overall performance of RAST and Voronoi is shown in the histograms in
%Figure~\ref{fig:winder-fullpage-rast} and \ref{fig:winder-fullpage-vor}. The figures
%show the performance of RAST and Voronoi on the full pages in the Winder
%data set. The RAST results are slightly better than the Voronoi. The Winder
%data set pages are well formed along Manhattan layout. The RAST algorithm
%excels at primarily vertical/horizontal layouts which might explain the better
%RAST results.

%  All Winder Full Page Images
%\begin{figure*}[fullpage-results]
%\begin{center}
%\caption{Full page results}
%\label{fig:fullpage-results}
% {
%  \subfigure[Winder full page RAST histogram]{ \includegraphics[width=6cm]{images/winder_fullpage_rast.png}}
%  \subfigure[Winder full page Voronoi histogram]{ \includegraphics[width=6cm]{images/winder_fullpage_vor.png}}\\
%  \subfigure[UW-III full page RAST histogram]{ \includegraphics[width=6cm]{images/uwiii_fullpage_rast_nonzero}}
%  \subfigure[UW-III full page Voronoi histogram]{ \includegraphics[width=6cm]{images/uwiii_fullpage_vor_nonzero}}
% }
%\end{center}
%\end{figure*}

%Given the full page results, we examine the results of the 300 row sliding
%window. The Voronoi segmentation is run on each of the 330 sub-images in each of
%the 459 Winder images. A histogram of the non-zero performance is displayed in
%Figure~\ref{fig:300-winder-vor}. 

%The RAST segmentation has slightly better results as shown in the histogram of
%non-zeroes in Figure~\ref{fig:300-winder-rast}. There is a large peak at the
%1.0 (perfect match) and several good peaks at ~0.70 and ~0.65. The RAST
%algorithm seems to perform better than Voronoi on the Winder data set strip
%images. 

%  All Winder 300 Strip Images
%\begin{figure}[300-winder-rast]
%\includegraphics[width=6.0cm]{images/300_winder_rast.png}
%\caption{Winder 300 strips RAST histogram}
%\label{fig:300-winder-rast}
%\end{figure}
%\begin{figure}[300-winder-vor]
%\includegraphics[width=6.0cm]{images/300_winder_vor.png}
%\caption{Winder 300 strips Voronoi histogram}
%\label{fig:300-winder-vor}
%\end{figure}

%We now more closely examine the behavior of RAST on the Winder 300 strips.
%\ref{fig:winder-300-double-2col300} shows a double column, all
%text page from \cite{winder2010extending} data set. The traversal of RAST
%segmentation down the page is shown in the jagged lines. The mean of RAST
%results down the page is 0.78 while the full page metric is 1.0.  The 0.78
%result is quite useful with many 1.0 hits. The zeroes at the bottom of the
%image (right side of the graph) are the bottom white space of the page. (No
%matches and no ground truth results in a zero.)

%\begin{figure}[winder-300-double-2col300]
%\includegraphics[width=6.0cm]{winder_full_rast_double_2col300_1.png}
%\caption{Winder 300 Strips vs Full Page RAST Two Columns}
%\label{fig:winder-300-double-2col300}
%\end{figure}

%Figure~\ref{fig:winder-300-double-pic-sci-2col300} shows a the results for a
%complicated page image. The top half of the page is a block diagram containing
%lines, squares, and enclosed text. The lack of data along the top of the page
%(left side of the graph) is indicative of the page segmentation algorithms to
%find text.

%\begin{figure}[winder-300-double-pic-sci-2col300]
%\includegraphics[width=6.0cm]{winder_full_rast_double_sci_2col300_3.png}
%\caption{Winder 300 Strips vs Full Page RAST Two Columns Scientific Journal Paper}
%\label{fig:winder-300-double-pic-sci-2col300}
%\end{figure}

%
% Discussion of UW-III Results
%
%\subsection{UW-III Data Results}
%
%The \cite{winder2010extending} data set is a relatively clean set of images. The results for
%those images is quite good. The UW-III data set is a much more challenging set
%of images. Many of the images contain strong vertical black lines that result
%from scanning a book (the edges of the pages are caught as shadows). 

%Figure~\ref{fig:uwiii-fullpage-rast-nonzero} shows a histogram of the RAST fullpage results of
%all 1600 UW-III images.  Figure~\ref{fig:uwiii-fullpage-vor-nonzero} shows the Voronoi
%fullpage results. Note there were several zero results not shown. Many of the
%UW-III images had strong black vertical artifacts indicative of book scanning
%Voronoi seems to have the edge in the fullpage analysis likely due to the
%images' scanning artifacts  (page edges). Neither RAST nor Voronoi page
%segmentation programs attempted to compensate for such artifacts. Filtering out
%the failures in further future work will require better preprocessing on those
%images to compensate for such artifacts. Note RAST seems to perform better
%overall than Voronoi.


% UW-III Full Page 
%\begin{figure}[uwiii-fullpage-rast]
%\includegraphics[width=6.0cm]{uwiii_fullpage_rast}
%\caption{UW-III Full Page RAST}
%\label{fig:uwiii-fullpage-rast}
%\end{figure}

%\begin{figure}[uwiii-fullpage-vor]
%\includegraphics[width=6.0cm]{uwiii_fullpage_vor}
%\caption{UW-III Full Page Voronoi}
%\label{fig:uwiii-fullpage-vor}
%\end{figure}


% UW-III 300 Strip
%\begin{figure}[300-uwiii-rast]
%\includegraphics[width=6.0cm]{uwiii_300_rast}
%\caption{UW-III 300 Row Strip RAST}
%\label{fig:300-uwiii-rast}
%\end{figure}

%\begin{figure}[300-uwiii-vor]
%\includegraphics[width=6.0cm]{uwiii_300_vor}
%\caption{UW-III 300 Row Strip Voronoi}
%\label{fig:300-uwiii-vor}
%\end{figure}

%\begin{figure}[uwiii-class-rast-vs-vor-300]
%\includegraphics[width=6.0cm]{uwiii_class_rast_vs_vor_300.png}
%\caption{UW-III 300 RAST vs Voronoi by Class}
%\label{fig:uwiii-class-rast-vs-vor-300}
%\end{figure}

% UW-III 300 Strip with Zeros Removed
%\begin{figure}[300_uwiii-rast]
%\includegraphics[width=6.0cm]{uwiii_300_rast_nonzero}
%\caption{UW-III 300 Row Strip RAST with Zeros Removed}
%\label{fig:300-uwiii-rast-nonzero}
%\end{figure}
%
%\begin{figure}[300_uwiii-vor]
%\includegraphics[width=6.0cm]{uwiii_300_vor_nonzero}
%\caption{UW-III 300 Row Strip Voronoi with Zeros Removed}
%\label{fig:300-uwiii-vor-nonzero}
%\end{figure}


% An example of a floating figure using the graphicx package.
% Note that \label must occur AFTER (or within) \caption.
% For figures, \caption should occur after the \includegraphics.
% Note that IEEEtran v1.7 and later has special internal code that
% is designed to preserve the operation of \label within \caption
% even when the captionsoff option is in effect. However, because
% of issues like this, it may be the safest practice to put all your
% \label just after \caption rather than within \caption{}.
%
% Reminder: the "draftcls" or "draftclsnofoot", not "draft", class
% option should be used if it is desired that the figures are to be
% displayed while in draft mode.
%
%\begin{figure}[!t]
%\centering
%\includegraphics[width=2.5in]{myfigure}
% where an .png filename suffix will be assumed under latex, 
% and a .pdf suffix will be assumed for pdflatex; or what has been declared
% via \DeclareGraphicsExtensions.
%\caption{Simulation Results}
%\label{fig_sim}
%\end{figure}

% Note that IEEE typically puts floats only at the top, even when this
% results in a large percentage of a column being occupied by floats.


% An example of a double column floating figure using two subfigures.
% (The subfig.sty package must be loaded for this to work.)
% The subfigure \label commands are set within each subfloat command, the
% \label for the overall figure must come after \caption.
% \hfil must be used as a separator to get equal spacing.
% The subfigure.sty package works much the same way, except \subfigure is
% used instead of \subfloat.
%
%\begin{figure*}[!t]
%\centerline{\subfloat[Case I]\includegraphics[width=2.5in]{subfigcase1}%
%\label{fig_first_case}}
%\hfil
%\subfloat[Case II]{\includegraphics[width=2.5in]{subfigcase2}%
%\label{fig_second_case}}}
%\caption{Simulation results}
%\label{fig_sim}
%\end{figure*}
%
% Note that often IEEE papers with subfigures do not employ subfigure
% captions (using the optional argument to \subfloat), but instead will
% reference/describe all of them (a), (b), etc., within the main caption.


% An example of a floating table. Note that, for IEEE style tables, the 
% \caption command should come BEFORE the table. Table text will default to
% \footnotesize as IEEE normally uses this smaller font for tables.
% The \label must come after \caption as always.
%
%\begin{table}[!t]
%% increase table row spacing, adjust to taste
%\renewcommand{\arraystretch}{1.3}
% if using array.sty, it might be a good idea to tweak the value of
% \extrarowheight as needed to properly center the text within the cells
%\caption{An Example of a Table}
%\label{table_example}
%\centering
%% Some packages, such as MDW tools, offer better commands for making tables
%% than the plain LaTeX2e tabular which is used here.
%\begin{tabular}{|c||c|}
%\hline
%One & Two\\
%\hline
%Three & Four\\
%\hline
%\end{tabular}
%\end{table}


% Note that IEEE does not put floats in the very first column - or typically
% anywhere on the first page for that matter. Also, in-text middle ("here")
% positioning is not used. Most IEEE journals/conferences use top floats
% exclusively. Note that, LaTeX2e, unlike IEEE journals/conferences, places
% footnotes above bottom floats. This can be corrected via the \fnbelowfloat
% command of the stfloats package.


% ========================================
%  Conclusion
% ========================================

\section{Conclusion}

In this paper, we studied the relative performance of the RAST and Voronoi page
segmentation algorithms when the images were constrained to 300 or 600 row
strips. The tests were run against the UW-III LINEWORD images and an additional
image test set from \cite{winder2010extending}. The fullpage images were divided
into smaller images using a sliding window to partially mimic a desktop
scanner.

The RAST results were almost uniformly better than Voronoi. Overall, RAST with
a 300 stripsize performed so well that the algorithm is a good candidate
for for implementation. The eventual goal would be to implement RAST in a
hardware circuit (ASIC) for real-time performance in an MFP.  The RAST 600
stripsize had little improvement over the 300 stripsize even in the cases where
the full image performed much better than the strip. The one exception was the
highly mixed layout from the Magazine images in \cite{winder2010extending}.

Neither the RAST nor the Voronoi program performed any preprocessing on the incoming
image other than binarization. The input to the segmentation is the raw source
image. There is no attempt to deskew or eliminate page margin issues. The more
favorable results with the Winder dataset (compared to the UW-III dataset) is
because the Winder dataset is relatively clean. The UW-III dataset has many
artifacts that would be unlikely in a modern desktop scanner, especially the
large black areas caused by catching multiple page edges in a book scan. 

%Most
%modern desktop scanners use a CIS (Compact Image Sensor) which has a very short
%depth of field. Older scanners used a reduction optics array (several mirrors
%which shrink and redirect the image onto a single small sensor). The
%reduction optics array scanners have a very high depth of field and will thus
%capture more extraneous information.

Future work will focus on preprocessing and image clean-up suitable for a
strip. The heavy black areas (book page edges) so indicative of the UW-III
dataset would be targeted by left/right margin clean-up using connected
components. The connected components larger than a specific size and 
stretched top to bottom of the strip could be ignored.

% conference papers do not normally have an appendix


% use section* for acknowledgement
\section*{Acknowledgment}

The authors would like to thank...

Amy Winder for her thesis, her code, and her time in explaining both.  Dr.
David Doermann for the UW-III Data Set CDROM. Python, NumPy
\cite{oliphant-2006-guide}, and Matplotlib
\cite{Hunter:2007} for making life easier.

% trigger a \newpage just before the given reference
% number - used to balance the columns on the last page
% adjust value as needed - may need to be readjusted if
% the document is modified later
%\IEEEtriggeratref{8}
% The "triggered" command can be changed if desired:
%\IEEEtriggercmd{\enlargethispage{-5in}}

% references section

% can use a bibliography generated by BibTeX as a .bbl file
% BibTeX documentation can be easily obtained at:
% http://www.ctan.org/tex-archive/biblio/bibtex/contrib/doc/
% The IEEEtran BibTeX style support page is at:
% http://www.michaelshell.org/tex/ieeetran/bibtex/
%\bibliographystyle{IEEEtran}
% argument is your BibTeX string definitions and bibliography database(s)
%\bibliography{IEEEabrv,../bib/paper}
%
% <OR> manually copy in the resultant .bbl file
% set second argument of \begin to the number of references
% (used to reserve space for the reference number labels box)
% \begin{thebibliography}{1}
% 
% \bibitem{IEEEhowto:kopka}
% H.~Kopka and P.~W. Daly, \emph{A Guide to \LaTeX}, 3rd~ed.\hskip 1em plus
%   0.5em minus 0.4em\relax Harlow, England: Addison-Wesley, 1999.
% 
% \bibitem{winder2010extending}
% Amy~Winder Thesis. 2008. 
% 
% \bibitem{IEEEhowto:Gatos}
% B.Gatos. ICDAR2007 Handwriting Segmentation Contest.
% 
% \bibitem{IEEEhowto:Philips}
% Ihsin Philips. Emperical Performance Evaluation of Graphics Recognition Systems. 1999. 
% 
% \end{thebibliography}

% The following two commands are all you need in the
% initial runs of your .tex file to
% produce the bibliography for the citations in your paper.
\bibliographystyle{abbrv}
\bibliography{strip_seg}  % sigproc.bib is the name of the Bibliography in this case
% You must have a proper ".bib" file
%  and remember to run:
% latex bibtex latex latex
% to resolve all references


% that's all folks
\end{document}

